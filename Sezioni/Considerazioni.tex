\documentclass[../Relezione.tex]{subfiles}

\begin{document}
\section{Considerazioni}
    \subsection{Nome del sito}
        Solitamente il nome di un sito internet incide in media tra il 10 ed il 20 percento nella sua qualità e impatto sull'utente. 
        L'utente potrebbe trovarlo troppo lungo e di difficile memorizzazione sopratutto perchè storpia la parola italiana "Fabrica" "in Fabrika"
        Nonostante ciò bisogna ricordare che \name\ è un azienda nata molto prima dell'avvento di internet e quindi mettere un altro nome al sito oppure addirittura cambiar solo nome per questo motivo il risultato sarebbe molto più drammatico. 
        Ha anche dei punti positivi il nome \name\ cioè nel ricordare che è un made in italy nel nome.
    \subsection{Considerazioni generali}
        \subsubsection{Testo}
            Bisogna far 2 divisioni: i titoli e il resto del testo.
            \begin{itemize}
            \item{\textbf{Titoli:}} la maggior parte delle volte troppo grandi, non danno informazioni aggiuntive o riassuntive ed essendo posti in luoghi in cui l'utente da una maggior attenzione tolgono spazio alle reali informazioni.
            \item{\textbf{Resto del testo:}} In generale di dimensione adatta ma non fa uso di keyword in grassetto per dare una informazione sommaria in mochissimo tempo, cosa molto amata dagli utenti.
            \end{itemize}

        \subsubsection{Design}
            Grafica molto pulita e leggera con colori molto contrastanti con nessuna parte poco visibile.
            Il sito in generale ha foto abbastanza pesanti ma giustificate dal fatto che è un sito vetrina ma complessivamente veloce da caricare.

        \subsubsection{Menu}
            Il menu come impostazione e posizione non ha nulla da segnalare però ha un fastidioso problema nella lingua, cioè le sezioni sono in lingua inglese anche dopo aver impostato la lingua italiana.
            Sucessivamente ho trovato poco comprensibile la differenza tra le sezioni "kitchen","details" e "houses", sarebbero state da dividere meglio.

        
        \subsubsection{Apertura finestre}
            Il sito apre sempre le immagini nella scheda in cui è stata cliccata l'immagine e in generale non vengono mai aperte nuove finestre.
            L'unico posto in cui il sito apre nuove schede è quando si clicca in una news nella sezione "News", ma questo è lecito visto che la notizia è sempre sviluppata in un PDF esterno.
        
        \subsubsection{Link}
           I link sono raramente presenti in questo sito, gli unici link sono quelli nella pagina contatti e news che non hanno la standardizzazione del carattere e la sottolineatura.
        
        
        \subsubsection{Ricerca}
          La ricerca non è presente ma non è un grosso problema visto che la quantità di informazioni presenti in questo sito è molto ridotta.

        \subsubsection{Back-button}
            E' perfettamente funzionante, si può sempre tornare ad una pagina precedente senza che vengano aperte nuove finistre o schede. 
        \subsubsection{Pagina 404}
            Se si prova ad andare su una pagina non prevista o non esistente, si viene correttamente indirizzati ad una pagina d'errore 404, in questa pagina è presenta un collegamento per tornare alla pagina home ma non è presente un aiuto alla ricerca.
\end{document}
